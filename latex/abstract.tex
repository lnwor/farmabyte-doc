\newpage
\pagenumbering{arabic}
\section{Abstract}
Il progetto riguarda la creazione di un applicativo software per la gestione interna dell’inventario delle farmacie e per l'interfacciamento con i clienti. 
Il cliente ha la possibilità di cercare quale sia la farmacia più vicina ad avere un certo medicinale e, dopo essersi autenticato, può inviare una prenotazione del farmaco.
La farmacia quindi può ricevere la prenotazione del cliente registrato al servizio che verrà poi finalizzata in acquisto. 
Inoltre la gestione dell’inventario del negozio permette alle farmacie di controllare l’elenco dei farmaci disponibili e ne facilita la gestione degli ordini e delle rimanenze, segnalando all'operatore le medicine in scadenza e in esaurimento.
L'applicativo fornisce anche una funzionalità di ricerca farmaci, per i clienti che desiderano verificare la disponibilità di un farmaco in una certa località.
