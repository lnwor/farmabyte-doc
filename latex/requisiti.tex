\newpage
\section{Documento dei requisiti}
\subsection{Raccolta dei requisiti}
\begin{itemize}
    \item[-] I clienti delle farmacie hanno a disposizione due servizi: controllare se un farmaco è disponibile vicino alla loro posizione e/o prenotarlo.
    \item[-] Il cliente fornisce la sua posizione che l'applicativo userà per indicargli le farmacie più vicine. Contemporaneamente specificherà il farmaco da cercare e l'applicativo fornirà le 10 farmacie più vicine ad averlo in magazzino indicandone se è disponibile o sta per terminare.
    \item[-] Per la prenotazione è necessario possedere un account
    \item[-] La prenotazione sarà composta da uno o più farmaci, dalla farmacia, e dal giorno. 
    \item[-] L'account viene creato in due fasi:
        1. Registrazione con nome, cognome, password, data di nascita, email e codice fiscale
        2. Autenticazione di persona in farmacia
    \item[-] La email deve essere univoca, la password di almeno 8 caratteri, contentente almeno un numero e un carattere alfabetico.
    \item[-] Per l'autenticazione è necessario mostrare il tesserino sanitario per l'identificazione in farmacia.
    \item[-] Il cliente può vedere la lista delle sue prenotazioni in corso
    \item[-] Il farmacista vede le prenotazioni, i farmaci disponibili in negozio e viene segnalato riguardo ai farmaci in esaurimento
    \item[-] Il farmacista può confermare le prenotazioni andate a buon fine
    \item[-] Se alla fine della giornata un utente non si presenta allora l'evento viene registrato, per poi avvisare il farmacista che può eventualmente bloccare l'utente per 1 mese.
    \item[-] Il sistema sarà ovviamente distribuito e di natura client-server con la presenza di un database centrale dove memorizzare i dati
    \item[-] La gestione delle vendite, degli ordini e modifiche di magazzino è gestita da un altro software
    \item[-] Non va considerata la gestione dei dati del personale 
\end{itemize}
