\newpage
\section{Documento dei Requisiti}\label{documento-dei-requisiti}

\subsection{Raccolta dei requisiti}\label{raccolta-dei-requisiti}

\begin{itemize}
  
\itemsep1pt\parskip0pt\parsep0pt
\item
  bpos fasdfa I clienti delle farmacie hanno a disposizione due servizi: controllare
  se un farmaco è disponibile vicino alla loro posizione e/o prenotarlo.
\item
  Il cliente fornisce la sua posizione che l'applicativo userà per
  indicargli le farmacie più vicine. Contemporaneamente specificherà il
  farmaco da cercare e l'applicativo fornirà le 10 farmacie più vicine
  ad averlo in magazzino indicandone se è disponibile o sta per
  terminare.
\item
  Per la prenotazione è necessario possedere un account
\item
  La prenotazione sarà composta da uno o più farmaci, dalla farmacia, e
  dal giorno.
\item
  L'account viene creato in due fasi:

  \begin{enumerate}
  \def\labelenumi{\arabic{enumi}.}
  \itemsep1pt\parskip0pt\parsep0pt
  \item
    Registrazione con nome, cognome, password, data di nascita, email e
    codice fiscale
  \item
    Autenticazione di persona in farmacia
  \end{enumerate}
\item
  La email deve essere univoca, la password di almeno 8 caratteri,
  contentente almeno un numero e un carattere alfabetico.
\item
  Per l'autenticazione è necessario mostrare il tesserino sanitario per
  l'identificazione in farmacia.
\item
  Il cliente può vedere la lista delle sue prenotazioni in corso
\item
  Il farmacista vede le prenotazioni, i farmaci disponibili in negozio e
  viene segnalato riguardo ai farmaci in esaurimento
\item
  Il farmacista può confermare le prenotazioni andate a buon fine
\item
  Se alla fine della giornata un utente non si presenta allora l'evento
  viene registrato, per poi avvisare il farmacista che può eventualmente
  bloccare l'utente per 1 mese.
\item
  Il sistema sarà ovviamente distribuito e di natura client-server con
  la presenza di un database centrale dove memorizzare i dati
\item
  La gestione delle vendite, degli ordini e modifiche di magazzino è
  gestita da un altro software
\item
  Non va considerata la gestione dei dati del personale
\end{itemize}

\begin{center}\rule{3in}{0.4pt}\end{center}

\subsection{Tabella dei Requisiti}\label{tabella-dei-requisiti}

\begin{longtable}[c]{@{}lll@{}}
\toprule\addlinespace
ID & Requisiti & Tipo
\\\addlinespace
\midrule\endhead
R1F & Localizzazione delle farmacie più vicine in base al farmaco da
cercare & Funzionale
\\\addlinespace
R2F & Specifica del farmaco da cercare da parte dell'utente & Funzionale
\\\addlinespace
R3F & Presentazione delle farmacie che dispongono di un farmaco &
Funzionale
\\\addlinespace
R4F & Registrazione di un account tramite l'interfaccia web & Funzionale
\\\addlinespace
R5F & Attivazione dell'account con identificazione fisica dell'utente
con documento & Funzionale
\\\addlinespace
R6F & La prenotazione sarà composta da uno o più farmaci, dalla farmacia
e dal giorno & Funzionale
\\\addlinespace
R7F & Identificazione attraverso email univoca e password di almeno 8
caratteri, contentente almeno un carattere alfabetico e un carattere
numerico & Funzionale
\\\addlinespace
R8F & Visualizzazione delle prenotazioni del cliente & Funzionale
\\\addlinespace
R9F & Visualizzazione delle prenotazioni della farmacia & Funzionale
\\\addlinespace
R10F & Visualizzazione del numero dei farmaci disponibili & Funzionale
\\\addlinespace
R11F & Notifica dei farmaci in esaurimento o in scadenza & Funzionale
\\\addlinespace
R12F & Conferma della prenotazione andata a buon fine & Funzionale
\\\addlinespace
R13F & Notifica della mancata finzalizzazione in acquisto di una
prenotazione & Funzionale
\\\addlinespace
R14F & Blocco dell'utente che effettua troppe prenotazioni senza
presentarsi & Funzionale
\\\addlinespace
R15F & Verrà memorizzato il numero di prenotazioni andate a buon fine &
Funzionale
\\\addlinespace
R1NF & Velocità di memorizzazione dei dati & Non Funzionale
\\\addlinespace
R2NF & Velocità della ricerca dei dati & Non Funzionale
\\\addlinespace
R3NF & Semplicità dell'interfaccia & Non Funzionale
\\\addlinespace
R4NF & Un utente non può avere più di un account verificato & Non
Funzionale
\\\addlinespace
R5NF & la gestione delle vendite e ordini è gestita da un altro software
& Non Funzionale
\\\addlinespace
R6NF & Per prenotare l'utente deve essere registrato & Non Funzionale
\\\addlinespace
R7NF & La gestione delle vendite, degli ordini e modifiche di magazzino
è gestita da un altro software & Non Funzionale
\\\addlinespace
R8NF & Non va considerata la gestione dei dati del personale & Non
Funzionale
\\\addlinespace
\bottomrule
\end{longtable}

\subsection{Vocabolario}\label{vocabolario}

\begin{longtable}[c]{@{}lll@{}}
\toprule\addlinespace
Voce & Definizione & Sinonimi
\\\addlinespace
\midrule\endhead
Cliente & Persona che usufruisce del servizio lato cliente & Utente
\\\addlinespace
ClienteRegistrato & Cliente che possiede un account identificato con cui
può effettuare prenotazioni & UtenteRegistrato
\\\addlinespace
Farmacia & Farmacia che aderisce al servizio & Punto vendita
\\\addlinespace
Farmaco & Medicinale che viene venduto in farmacia
\\\addlinespace
Farmacista & Utente che accede con le credenziali della farmacia &
Operatore
\\\addlinespace
Prenotazione & Richiesta di farmaci da comprare in negozio
\\\addlinespace
Data e ora prenotazione & Indicazione temporale del momento in cui
avverrà la prenotazione
\\\addlinespace
Posizione & Luogo della ricerca o collocamento geografico della farmacia
\\\addlinespace
Credenziali & Insieme composto da email e password necessari per
acccedere al sistema
\\\addlinespace
Email & Indirizzo di posta elettronica del cliente utilizzata anche per
l'autenticazione
\\\addlinespace
Password & Codice alfanumerico di almeno 8 caratteri
\\\addlinespace
Magazzino & Luogo fisico in cui vengono conservati i farmaci di un punto
vendita & Deposito
\\\addlinespace
\bottomrule
\end{longtable}

\begin{center}\rule{3in}{0.4pt}\end{center}

\subsection{Scenari}\label{scenari}

\begin{longtable}[c]{@{}ll@{}}
\toprule\addlinespace
\textbf{Titolo} & GestioneFarmacia
\\\addlinespace
\textbf{Descrizione} & Gestione dell'utenza di un cliente registrato
\\\addlinespace
\textbf{Attori} & Farmacista
\\\addlinespace
\textbf{Relazioni} & Login, SospensioneUtenza, VerificaIdentità,
ResocontoFarmaci, ResocontoPrenotazioni
\\\addlinespace
\textbf{Precondizioni} &
\\\addlinespace
\textbf{Postcondizioni} &
\\\addlinespace
\textbf{Scenario Principale} & 1. Login 2. Il farmacista può eseguire la
verifica, sospendere un'account, controllare le prenotazioni e i farmaci
in magazzino
\\\addlinespace
\textbf{Scenari Alternativi} &
\\\addlinespace
\textbf{Requisiti non funzionali} & Velocità di ricerca dei dati e
semplicità di navigazione tra le diverse maschere
\\\addlinespace
\textbf{Punti aperti} &
\\\addlinespace
\bottomrule
\end{longtable}

\begin{longtable}[c]{@{}ll@{}}
\toprule\addlinespace
\textbf{Titolo} & ResocontoUtenti
\\\addlinespace
\textbf{Descrizione} & Si controllano gli utenti con potenzialmente
sospendibili
\\\addlinespace
\textbf{Attori} & Farmacista
\\\addlinespace
\textbf{Relazioni} & SospensioneUtenza, GestioneFarmacia
\\\addlinespace
\textbf{Precondizioni} &
\\\addlinespace
\textbf{Postcondizioni} & Viene mostrato l'elenco degli Utenti a rischio
sospensione
\\\addlinespace
\textbf{Scenario Principale} & 1. Il Farmacista va nella schermata di
visualizzazione utenti 2. Il sistema recupera l'elenco degli utenti a
rischio 3. Il sistema mostra a video l'elenco degli utenti
\\\addlinespace
\textbf{Scenari Alternativi} &
\\\addlinespace
\textbf{Requisiti non funzionali} & Velocità di ricerca dei dati e
semplicità di navigazione tra le diverse maschere
\\\addlinespace
\textbf{Punti aperti} &
\\\addlinespace
\bottomrule
\end{longtable}

\begin{longtable}[c]{@{}ll@{}}
\toprule\addlinespace
\textbf{Titolo} & ResocontoFarmaci
\\\addlinespace
\textbf{Descrizione} & Viene mostrato l'elenco dei farmaci in scadenza o
in esaurimento
\\\addlinespace
\textbf{Attori} & Farmacista
\\\addlinespace
\textbf{Relazioni} & GestioneFarmacia
\\\addlinespace
\textbf{Precondizioni} &
\\\addlinespace
\textbf{Postcondizioni} & Viene mostrato l'elenco degli Utenti a rischio
sospensione
\\\addlinespace
\textbf{Scenario Principale} & 1. Il Farmacista va nella schermata di
visualizzazione farmaci 2. Il sistema recupera l'elenco dei farmaci in
esaurimento o in scadenza 3. Il sistema mostra a video l'elenco
richiesto
\\\addlinespace
\textbf{Scenari Alternativi} &
\\\addlinespace
\textbf{Requisiti non funzionali} & Velocità di ricerca dei dati e
semplicità di navigazione tra le diverse maschere
\\\addlinespace
\textbf{Punti aperti} &
\\\addlinespace
\bottomrule
\end{longtable}

\begin{longtable}[c]{@{}ll@{}}
\toprule\addlinespace
\textbf{Titolo} & ControlloPrenotazioni
\\\addlinespace
\textbf{Descrizione} & Si controllano le prenotazioni non terminate
\\\addlinespace
\textbf{Attori} & Farmacista
\\\addlinespace
\textbf{Relazioni} & ConfermaPrenotazione,GestioneFarmacia
\\\addlinespace
\textbf{Precondizioni} &
\\\addlinespace
\textbf{Postcondizioni} & Viene mostrato l'elenco delle prenotazioni
\\\addlinespace
\textbf{Scenario principale} & 1. Il Farmacista va nella schermata di
visualizzazione prenotazioni 2. Il sistema recupera l'elenco delle
prenotazioni giornaliere 3. Il sistema mostra a video l'elenco delle
prenotazioni
\\\addlinespace
\textbf{Scenari Alternativi} &
\\\addlinespace
\textbf{Requisiti non funzionali} & Velocità di ricerca dei dati
\\\addlinespace
\textbf{Punti aperti} &
\\\addlinespace
\bottomrule
\end{longtable}

\begin{longtable}[c]{@{}ll@{}}
\toprule\addlinespace
\textbf{Titolo} & ConfermaPrenotazione
\\\addlinespace
\textbf{Descrizione} & Il Farmacista conferma la prenotazione avvenuta
\\\addlinespace
\textbf{Attori} & Farmacista
\\\addlinespace
\textbf{Relazioni} & ControlloPrenotazioni
\\\addlinespace
\textbf{Precondizioni} &
\\\addlinespace
\textbf{Postcondizioni} & La prenotazione viene confermata
\\\addlinespace
\textbf{Scenario principale} & 1. ControlloPrenotazioni 2. Il Farmacista
conferma l'avvenuta prenotazione
\\\addlinespace
\textbf{Scenari Alternativi} &
\\\addlinespace
\textbf{Requisiti non funzionali} & Semplicità dell'interfaccia
\\\addlinespace
\textbf{Punti aperti} &
\\\addlinespace
\bottomrule
\end{longtable}

\begin{longtable}[c]{@{}ll@{}}
\toprule\addlinespace
\textbf{Titolo} & Registrazione
\\\addlinespace
\textbf{Descrizione} & Il cliente si registra al servizio
\\\addlinespace
\textbf{Attori} & Cliente
\\\addlinespace
\textbf{Relazioni} &
\\\addlinespace
\textbf{Precondizioni} & Il cliente dispone di un codice fiscale valido
\\\addlinespace
\textbf{Postcondizioni} & Il cliente è registrato nel sistema ed è posto
in attesa della verifica
\\\addlinespace
\textbf{Scenario principale} & 1. Il cliente accede alla sezione di
registrazione 2. Il cliente inserisce i propri dati: nome, cognome, data
di nascita, email e il codice fiscale 3. Il cliente termina la
registrazione, se avvenuta con successo gli viene mostrata la conferma e
viene reindirizzato alla pagina principale
\\\addlinespace
\textbf{Scenari Alternativi} & Scenario a: il codice fiscale è già
registrato 3. Il sistema verifica che è già presente un utente con quel
codice fiscale, quindi notifica il cliente con un messaggio di errore.
\\\addlinespace
\textbf{Requisiti non funzionali} & Semplicità dell'interfaccia
\\\addlinespace
\textbf{Punti aperti} &
\\\addlinespace
\bottomrule
\end{longtable}

\begin{longtable}[c]{@{}ll@{}}
\toprule\addlinespace
\textbf{Titolo} & VerificaIdentità
\\\addlinespace
\textbf{Descrizione} & Verifica dell'identità dell'utente registrato
\\\addlinespace
\textbf{Attori} & Cliente, Farmacista
\\\addlinespace
\textbf{Relazioni} & GestioneFarmacia
\\\addlinespace
\textbf{Precondizioni} & Il cliente è registrato
\\\addlinespace
\textbf{Postcondizioni} & L'utente è stato verificato e il suo account
viene abilitato per effettuare delle prenotazioni
\\\addlinespace
\textbf{Scenario principale} & 1. Il cliente va in farmacia con il
documento specificato in fase di registrazione 2. Il cliente viene
identificato dal farmacista 3. Il farmacista chiede al sistema di
recuperare l'utente 4. Il farmacista attiva l'account dell'utente
\\\addlinespace
\textbf{Scenari alternativi} & 4. Il sistema non trova nessun utente,
segnala il farmacista
\\\addlinespace
\textbf{Requisiti non funzionali} & Velocità di memorizzazione e
semplicità di navigazione tra le diverse maschere
\\\addlinespace
\textbf{Punti aperti} &
\\\addlinespace
\bottomrule
\end{longtable}

\begin{longtable}[c]{@{}ll@{}}
\toprule\addlinespace
\textbf{Titolo} & SospensioneUtenza
\\\addlinespace
\textbf{Descrizione} & Se un utente non ha concluso troppe prenotazioni
allora viene proposta la sospensione dell'utente al farmacista
\\\addlinespace
\textbf{Attori} & Farmacista
\\\addlinespace
\textbf{Relazioni} & ResocontoPrenotazioni, GestioneFarmacia
\\\addlinespace
\textbf{Precondizioni} & Il cliente è diffidato dal sistema (ha molte
prenotazioni non concluse)
\\\addlinespace
\textbf{Postcondizioni} & Il cliente non può più effettuare prenotazioni
per 30 giorni
\\\addlinespace
\textbf{Scenario principale} & 1. Si verifica l'evento FineGiornata 2.
Il Sistema recupera l'elenco delle prenotazioni e lo analizza 3. Se il
Sistema rileva un numero eccessivo di prenotazioni non concluse per un
determinato utente, allora lo segnala al farmacista 4. Il farmacista, se
ritiene necessario, può confermare la sospensione dell'utente
\\\addlinespace
\textbf{Scenari alternativi} &
\\\addlinespace
\textbf{Requisiti non funzionali} & Velocità nella ricerca dei dati e
semplicità dell'interfaccia
\\\addlinespace
\textbf{Punti aperti} &
\\\addlinespace
\bottomrule
\end{longtable}

\begin{longtable}[c]{@{}ll@{}}
\toprule\addlinespace
\textbf{Titolo} & ResocontoPrenotazioni
\\\addlinespace
\textbf{Descrizione} & Si controllano le prenotazioni non terminate per
utente
\\\addlinespace
\textbf{Attori} & Farmacista, FineGiornata
\\\addlinespace
\textbf{Relazioni} & Resoconto, SospensioneUtenza, GestioneFarmacia
\\\addlinespace
\textbf{Precondizioni} &
\\\addlinespace
\textbf{Postcondizioni} & Viene mostrato l'elenco delle prenotazioni
\\\addlinespace
\textbf{Scenario principale} & 1. Il Farmacista va nella schermata di
visualizzazione farmaci 2. Il sistema recupera l'elenco delle
prenotazioni giornaliere 3. Il sistema mostra a video l'elenco delle
prenotazioni
\\\addlinespace
\textbf{Scenari Alternativi} &
\\\addlinespace
\textbf{Requisiti non funzionali} & Velocità di ricerca dei dati e
semplicità di navigazione tra le diverse maschere
\\\addlinespace
\textbf{Punti aperti} &
\\\addlinespace
\bottomrule
\end{longtable}

\begin{longtable}[c]{@{}ll@{}}
\toprule\addlinespace
\textbf{Titolo} & RicercaFarmaci
\\\addlinespace
\textbf{Descrizione} & L'utente verifica la disponibilità di un
particolare farmaco nelle farmacie più vicine a lui
\\\addlinespace
\textbf{Attori} & Cliente
\\\addlinespace
\textbf{Relazioni} &
\\\addlinespace
\textbf{Precondizioni} &
\\\addlinespace
\textbf{Postcondizioni} & Si visualizza la lista delle farmacie con
disponibilità
\\\addlinespace
\textbf{Scenario principale} & 1. Il cliente si reca nella pagina di
ricerca 2. Il cliente inserisce il nome del farmaco per cui eseguire la
ricerca 3. Il sistema ottiene la lista delle farmacie aventi il farmaco
specificato entro un range dalla località specificata. 4. Il sistema
ordina la lista in base alla distanza geografica dalla zona dell'utente,
in ordine crescente 5. La lista viene mostrata all'utente
\\\addlinespace
\textbf{Scenari Alternativi} & Scenario alternativo A: 1. Il cliente
registrato effettua il login 2. Il cliente si reca nella pagina di
ricerca 3. Il cliente inserisce il nome del farmaco per cui eseguire la
ricerca 4. Il sistema ottiene la lista delle farmacie aventi il farmaco
specificato entro un range dalla località specificata. 5. Il sistema
ordina la lista in base alla distanza geografica dalla zona dell'utente,
in ordine crescente 6. La lista viene mostrata all'utente 7. L'utente
può selezionare il farmaco per cominciare una prenotazione di quel
farmaco nella farmacia scelta Scenario alternativo B: 1. Il cliente
registrato effettua il login 2. Il cliente avvia una nuova prenotazione
dalla home 3. Il cliente seleziona la farmacia in cui effettuare la
prenotazione 4. Il cliente inserisce il nome del farmaco per cui
eseguire la ricerca 5. Il sistema ottiene la la lista dei farmaci
disponibili nella farmacia scelta il cui nome inizia per il testo
inserito dall'utente 6. La lista viene mostrata all'utente 7. L'utente
può selezionare il farmaco corrispondente, se è presente nella lista, ed
aggiungerlo alla prenotazione in corso
\\\addlinespace
\textbf{Requisiti non funzionali} & Velocità di ricerca dei dati e
semplicità di navigazione tra le diverse maschere
\\\addlinespace
\textbf{Punti aperti} &
\\\addlinespace
\bottomrule
\end{longtable}

\begin{longtable}[c]{@{}ll@{}}
\toprule\addlinespace
\textbf{Titolo} & NuovaPrenotazione
\\\addlinespace
\textbf{Descrizione} & L'utente prenota a suo nome una lista di farmaci
\\\addlinespace
\textbf{Attori} & ClienteRegistrato
\\\addlinespace
\textbf{Relazioni} & GestionePrenotazioni
\\\addlinespace
\textbf{Precondizioni} &
\\\addlinespace
\textbf{Postcondizioni} & Il sistema ha memorizzato i dati della
prenotazione, in attesa di conferma da parte della farmacia
\\\addlinespace
\textbf{Scenario principale} & 1. Il cliente esegue il \textbf{Login} 2.
Il cliente seleziona i farmaci che vuole prenotare, la quantità, e
inserisce la data di ritiro desiderata 3. Il cliente invia la richiesta
di prenotazione 4. Il sistema pone la richiesta in attesa di conferma
\\\addlinespace
\textbf{Scenari Alternativi} & Scenario a: La farmacia non dispone dei
farmaci richiesti. 4. Il sistema nota che la farmacia non ha
disponibilità di almeno uno dei farmaci specificati 5. Viene inviato al
cliente un messaggio di errore
\\\addlinespace
\textbf{Requisiti non funzionali} & Velocità di verifica dei dati e
semplicità di navigazione tra le diverse maschere
\\\addlinespace
\textbf{Punti aperti} &
\\\addlinespace
\bottomrule
\end{longtable}

\begin{longtable}[c]{@{}ll@{}}
\toprule\addlinespace
\textbf{Titolo} & ListaPrenotazioni
\\\addlinespace
\textbf{Descrizione} & L'utente ottiene la lista delle proprie
prenotazioni passate ed in corso
\\\addlinespace
\textbf{Attori} & ClienteRegistrato
\\\addlinespace
\textbf{Relazioni} & GestionePrenotazioni
\\\addlinespace
\textbf{Precondizioni} &
\\\addlinespace
\textbf{Postcondizioni} & Al cliente viene mostrata la lista delle
prenotazioni passate ed in corso
\\\addlinespace
\textbf{Scenario Principale} & 1. Il cliente esegue il Login 2. Il
cliente seleziona l'opzione di visualizzazione della lista delle
prenotazioni 3. Al cliente viene mostrato l'elenco delle prenotazioni
effettuate
\\\addlinespace
\textbf{Scenari Alternativi} &
\\\addlinespace
\textbf{Requisiti non funzionali} & Velocità di verifica dei dati e
semplicità di navigazione tra le diverse maschere
\\\addlinespace
\textbf{Punti aperti} &
\\\addlinespace
\bottomrule
\end{longtable}

\begin{longtable}[c]{@{}ll@{}}
\toprule\addlinespace
\textbf{Titolo} & GestionePrenotazioni
\\\addlinespace
\textbf{Descrizione} & Gestione delle prenotazioni di un cliente
registrato
\\\addlinespace
\textbf{Attori} & ClienteRegistrato
\\\addlinespace
\textbf{Relazioni} & Login, ListaPrenotazioni, NuovaPrenotazione
\\\addlinespace
\textbf{Precondizioni} &
\\\addlinespace
\textbf{Postcondizioni} &
\\\addlinespace
\textbf{Scenario Principale} & 1. Il cliente registrato esegue il login
2. Il cliente può visualizzare le proprie prenotazioni passate o in
corso e può effettuare nuove prenotazioni.
\\\addlinespace
\textbf{Scenari Alternativi} &
\\\addlinespace
\textbf{Requisiti non funzionali} & Velocità di verifica dei dati e
semplicità di navigazione tra le diverse maschere
\\\addlinespace
\textbf{Punti aperti} &
\\\addlinespace
\bottomrule
\end{longtable}

\begin{longtable}[c]{@{}ll@{}}
\toprule\addlinespace
\textbf{Titolo} & Login
\\\addlinespace
\textbf{Descrizione} & Permette di accedere al sistema
\\\addlinespace
\textbf{Attori} & ClienteRegistrato, Farmacista
\\\addlinespace
\textbf{Relazioni} & NuovaPrenotazione, GestioneFarmacia
\\\addlinespace
\textbf{Precondizioni} &
\\\addlinespace
\textbf{Postcondizioni} & L'utente ha accesso al sistema, limitato in
base ai suoi privilegi
\\\addlinespace
\textbf{Scenario principale} & 1. L'utente inserisce le credenziali di
accesso 2. Il sistema verifica le credenziali 3. Se le credenziali sono
corrette, viene presentata la schermata iniziale
\\\addlinespace
\textbf{Scenari Alternativi} & Scenario a: Credenziali non riconosciute.
3. Il sistema non riconosce le credenziali e rispedisce l'utente alla
schermata di login con un messaggio di errore
\\\addlinespace
\textbf{Requisiti non funzionali} & Velocità di verifica delle
credenziali
\\\addlinespace
\textbf{Punti aperti} &
\\\addlinespace
\bottomrule
\end{longtable}

\begin{longtable}[c]{@{}ll@{}}
\toprule\addlinespace
\textbf{Titolo} & AggiornamentoUtenti
\\\addlinespace
\textbf{Descrizione} & Aggiorna l'elenco degli utenti a rischio
sospensione
\\\addlinespace
\textbf{Attori} & FineGiornata
\\\addlinespace
\textbf{Relazioni} &
\\\addlinespace
\textbf{Precondizioni} &
\\\addlinespace
\textbf{Postcondizioni} & Il DataBase degli utenti è aggiornato
\\\addlinespace
\textbf{Scenario principale} & 1. Si verifica l'evento FineGiornata 2.
Il sistema controlla le prenotazioni non andate a buon fine 3. Il
sistema aggiorna i dati relativi alle infrazioni degli utenti
\\\addlinespace
\textbf{Scenari Alternativi} &
\\\addlinespace
\textbf{Requisiti non funzionali} & Velocità della ricerca dei dati
\\\addlinespace
\textbf{Punti aperti} &
\\\addlinespace
\bottomrule
\end{longtable}

\begin{longtable}[c]{@{}ll@{}}
\toprule\addlinespace
\textbf{Titolo} & Aggiornamento Farmaci
\\\addlinespace
\textbf{Descrizione} & Aggiorna l'elenco dei farmaci in magazzino
\\\addlinespace
\textbf{Attori} & ModificheFarmaci
\\\addlinespace
\textbf{Relazioni} &
\\\addlinespace
\textbf{Precondizioni} &
\\\addlinespace
\textbf{Postcondizioni} & Il DataBase dei farmaci è aggiornato
\\\addlinespace
\textbf{Scenario principale} & 1. Si verifica l'evento ModificheFarmaci
2. Il sistema recupera le modifiche dal DataBase Remoto 3. Il sistema
aggiorna i dati relativi ai farmaci in magazzino
\\\addlinespace
\textbf{Scenari Alternativi} &
\\\addlinespace
\textbf{Requisiti non funzionali} & Velocità della ricerca dei dati
\\\addlinespace
\textbf{Punti aperti} &
\\\addlinespace
\bottomrule
\end{longtable}

\begin{center}\rule{3in}{0.4pt}\end{center}

\subsection{Analisi del Rischio}\label{analisi-del-rischio}

\subsubsection{Tabella Valutazione dei
Beni}\label{tabella-valutazione-dei-beni}

\begin{longtable}[c]{@{}lll@{}}
\toprule\addlinespace
Bene & Valore & Esposizione
\\\addlinespace
\midrule\endhead
Sistema Informativo & Alto. Fondamentale per il funzionamento del
servizio & Alta. Perdita finanziaria e di immagine
\\\addlinespace
Informazioni dei clienti & Alto. Dati generali dei clienti della
farmacia, comprese le credenziali & Alta. Perdita di immagine dovuta
alla divulgazione di dati sensibili
\\\addlinespace
Informazioni relative al personale & Alto. Dati relativi ai farmacisti,
incluse le credenziali di accesso all'area riservata & Molto Alta.
Perdita finanziaria dovuta a usi impropri delle credenziali con
privilegi elevati. Perdita di immagine possibile con la divulgazione dei
dati relativi ai clienti
\\\addlinespace
Dati delle prenotazioni & Alto. Necessario per tenere traccia delle
prenotazioni & Molto Alta. Perdita finanziaria dovuta allo smarrimento
di prenotazioni. Perdita di immagine con la divulgazione dei farmaci
prenotati dai clienti
\\\addlinespace
\bottomrule
\end{longtable}

\subsubsection{Tabella
Minacce/Controlli}\label{tabella-minaccecontrolli}

\begin{longtable}[c]{@{}llll@{}}
\toprule\addlinespace
Minaccia & Probabilità & Controllo & Fattibilità
\\\addlinespace
\midrule\endhead
Furto credenziali Farmacista & Alta & Controllo sulla sicurezza della
password - Log delle operazioni & Costo implementativo molto basso
\\\addlinespace
Furto credenziali Cliente & Alta & Controllo sulla sicurezza della
password - Log delle operazioni & Costo implementativo molto basso
\\\addlinespace
Alterazione o intercettazione delle comunicazioni & Alta & Utilizzo di
un canale sicuro - Log delle operazioni & Basso costo di realizzazione
con determinati protocolli
\\\addlinespace
Accesso non autorizzato al database & Bassa & Accesso da macchine sicure
- Log di tutte le operazioni & Basso costo di realizzazione, il server
deve essere ben custodito
\\\addlinespace
DoS & Bassa & Controllo e limitazione delle richieste & Media
complessità di implementazione
\\\addlinespace
Saturazione del database & Bassa & 1. Limitazione delle richieste in un
dato intervallo di tempo. 2. Limite di tempo per la verifica di un
cliente & Media complessità di implementazione
\\\addlinespace
\bottomrule
\end{longtable}

\subsubsection{Analisi Tecnologica della
Sicurezza}\label{analisi-tecnologica-della-sicurezza}

\begin{longtable}[c]{@{}ll@{}}
\toprule\addlinespace
Tecnologia & Vulnerabilità
\\\addlinespace
\midrule\endhead
Autenticazione email/password & • Utente rivela volontariamente la
password Utente rivela la password con un attacco di ingegneria sociale
• Utente non esce dal sistema dopo aver eseguito le operazioni •
Password banali
\\\addlinespace
Cifratura comunicazioni & • In caso di cifratura simmetrica particolare
attenzione va alla lunghezza delle chiavi ed alla loro memorizzazione •
La memorizzazione è un fattore fondamentale anche nella cifratura
asimmetrica
\\\addlinespace
Architettura Client/Server & • DoS • Man in the Middle • Sniffing delle
comunicazioni
\\\addlinespace
\bottomrule
\end{longtable}

\subsubsection{Security Use Case \& Misuse Case
Scenari}\label{security-use-case-misuse-case-scenari}

\begin{longtable}[c]{@{}ll@{}}
\toprule\addlinespace
\textbf{Titolo} & Riservatezza
\\\addlinespace
\textbf{Descrizione} & I dati non sono accessibili da chi non ne ha i
permessi
\\\addlinespace
\textbf{Misuse case} & Sniffing
\\\addlinespace
\textbf{Relazioni} &
\\\addlinespace
\textbf{Precondizioni} & L'attaccante ha i mezzi per intercettare i
messaggi del sistema
\\\addlinespace
\textbf{Postcondizioni} & Il sistema impedisce all'attaccante di
decifrare (in tempi utili) i messaggi intercettati
\\\addlinespace
\textbf{Scenario principale} & 1. Il Sistema protegge i messaggi 2.
L'attaccante riesce ad intercettare un messaggio 3. L'attaccante prova a
decifrare i messaggi, ma non riesce a trovare un modo per farlo
abbastanza velocemente
\\\addlinespace
\textbf{Scenari di un attacco avvenuto con successo} & 1. Il Sistema
protegge i messaggi 2. L'attaccante riesce ad intercettare un messaggio
3. L'attaccante riesce a decifrare i messaggi e a leggerne il contenuto,
ma solamente per una sessione di un utente
\\\addlinespace
\bottomrule
\end{longtable}

\begin{longtable}[c]{@{}ll@{}}
\toprule\addlinespace
\textbf{Titolo} & Integrità
\\\addlinespace
\textbf{Descrizione} & Integrità dei dati del sistema
\\\addlinespace
\textbf{Misuse case} & ManInTheMiddle
\\\addlinespace
\textbf{Relazioni} &
\\\addlinespace
\textbf{Precondizioni} & 1. L'attaccante ha i mezzi per intercettare i
messaggi del sistema 2. L'attaccante ha i mezzi per modificare i
messaggi 3. L'attaccante ha i mezzi per spedire il messaggio modificato
al destinatario
\\\addlinespace
\textbf{Postcondizioni} & Il sistema rileva il messaggio contraffatto
\\\addlinespace
\textbf{Scenario principale} & 1. Il Sistema protegge i messaggi 2.
L'attaccante riesce ad intercettare un messaggio e lo modifica 3. Il
sistema si accorge del messaggio contraffatto e lo segna nei log
\\\addlinespace
\textbf{Scenari di un attacco avvenuto con successo} & 1. Il Sistema
protegge i messaggi 2. L'attaccante riesce ad intercettare un messaggio
e lo modifica 3. Il sistema accetta il messaggio e agisce di
conseguenza, segnando il messaggio nei log
\\\addlinespace
\bottomrule
\end{longtable}

\begin{longtable}[c]{@{}ll@{}}
\toprule\addlinespace
\textbf{Titolo} & ControlloAccessi
\\\addlinespace
\textbf{Descrizione} & L'accesso alle funzionalità del sistema deve
essere controllato
\\\addlinespace
\textbf{Misuse case} & FurtoCredenziali, ManInTheMiddle
\\\addlinespace
\textbf{Relazioni} &
\\\addlinespace
\textbf{Precondizioni} & L'attaccante ha i mezzi per carpire in tutto o
in parte le credenziali di accesso di un cliente o di un farmacista
\\\addlinespace
\textbf{Postcondizioni} & Il sistema blocca l'accesso non autorizzato e
notifica il tentativo di accesso
\\\addlinespace
\textbf{Scenario principale} & 1. L'attaccante tenta di accedere al
servizio spacciandosi per un utente legittimo, di cui conosce le
credenziali solo in parte (ad esempio mediante attacco con dizionario)
2. Il sistema non riconosce le credenziali, restituendo un errore 3. In
seguito ad un numero fissato di tentativi falliti, il sistema blocca
temporaneamente l'accesso a quell'utente e notifica l'anomalia a chi di
dovere
\\\addlinespace
\textbf{Scenari di un attacco avvenuto con successo} & 1. L'attaccante
riesce a carpire le credenziali di accesso complete di un utente in un
qualsiasi modo 2. Il sistema riconosce la correttezza delle credenziali,
e fornisce l'accesso al soggetto malevolo 3. L'attaccante ha libero
accesso al sistema, con privilegi diversi in base al tipo di utente
\\\addlinespace
\bottomrule
\end{longtable}

\subsubsection{Requisiti di Protezione dei
Dati}\label{requisiti-di-protezione-dei-dati}

Sussistono inoltre i seguenti requisiti: 1. I dati salvati devono essere
protetti da un attaccante che abbia accesso al sistema, prendendo misure
di sicurezza fisica, eventualmente cifrando i dati. 2. I dati inviati
tra le parti remote devono essere protetti, utilizzando la cifratura dei
dati. 3. Tutte le azioni avvenute sul sistema devono essere tracciate
tramite un sistema di log. La visione e l'analisi dei log verrà gestita
con un editor di testo esterno, accessibile solo al personale
autorizzato.

\begin{longtable}[c]{@{}lll@{}}
\toprule\addlinespace
ID & Requisiti & Tipo
\\\addlinespace
\midrule\endhead
R16F & Implementazione di un sistema di log per tracciare tutti i
messaggi tra i client e i server, inclusi gli accessi, le richieste di
prenotazione, di conferma, di sospensione e di invio e ricezione di dati
& Funzionale
\\\addlinespace
R9NF & I dati salvati devono essere protetti da un attaccante che abbia
accesso al sistema, prendendo misure di sicurezza fisica, eventualmente
cifrando i dati & Non Funzionale
\\\addlinespace
R10NF & I dati inviati tra le parti remote devono essere protetti,
utilizzando la cifratura dei dati & Non Funzionale
\\\addlinespace
\bottomrule
\end{longtable}
